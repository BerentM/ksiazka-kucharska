\documentclass{article}

\usepackage[T1]{fontenc}

\usepackage{graphicx}
\graphicspath{ {./obrazki/} }
\usepackage{wrapfig}
\usepackage[export]{adjustbox}
\usepackage{polski}
\usepackage[utf8]{inputenc}
\usepackage[polish]{babel}

\title{Pyszności}
\author{Kingi i Mateusza}
\date{2020-03-29}

\begin{document}
    \pagenumbering{gobble}
    \maketitle
    \newpage
    \pagenumbering{arabic}

    % #########################################################################
    % #########################################################################
    % #########################################################################
    \section{Dania obiadowe}
    \medskip
    \subsection{Tortilla}
    \bigskip
    \paragraph{Składniki:}
    \begin{wrapfigure}[1]{r}{0.4\textwidth}
        \includegraphics[width=0.4\textwidth]{tortilla.jpg}
    \end{wrapfigure}
    \begin{enumerate}
        \item 250g mąki (tortowa z Łasina)
        \item 1/2 szklanki gorącej wody
        \item 1/2 łyżeczki soli
        \item 2 łyżki smalcu
    \end{enumerate}

    \paragraph{Przygotowanie}
    \begin{enumerate}
        \item Mąkę przesiewamy do miski.
        \item Smalec rozpuszczamy w gorącej wodzie z dodatkiem soli.
        \item Wyrabiamy ciasto przez około 10 minut.
        \item Ciasto dzielimy na około 8-10 części i rozwałkujemy na stolnicy,
            do przezroczystości.
        \item Smażymy placki tortilli na suchej, teflonowej lub ceramicznej
            patelni po ok. 2-4 minuty z każdej storny, aż do uzyskania
            brązowych, rumianych plamek.
        \item Gorące placki nadziewamy farszem i zawijamy.
    \end{enumerate}

    \paragraph{Porada}
    Zimne placki tracą swoją elastyczność, przed zawijaniem należy podgrzać je
    na gorącej, suchej patelni.
    \newpage

    \subsection{Pancakes}
    \bigskip
    \paragraph{Składniki:}
    \begin{wrapfigure}[1]{r}{0.4\textwidth}
        \includegraphics[width=0.4\textwidth]{pancakes.jpg}
    \end{wrapfigure}
    \begin{enumerate}
        \item 2 szklanki mąki
        \item 2 jajka
        \item 1 i 1/2 szklanki mleka
        \item 75 g rozpuszczonego masła
        \item 3,5 łyżeczki proszku do pieczenia
        \item 1/3 szklanki cukru pudru/cukru brązowego
        \item szczypta soli
    \end{enumerate}

    \paragraph{Przygotowanie}
    \begin{enumerate}
        \item Mąkę przesiać.
        \item Jajka roztrzepać i wymieszać z mlekiem, następnie połączyć z
            pozostałymi składnikami, na końcu dodać masło.
        \item Odstawić na 15 minut.
        \item Smażyć na suchej patelni teflonowej, bez dodatkowego tłuszczu, na
            średnim ogniu, tak aby dać ciastu szansę na wyrośnięcie.
        \item Delikatnie przewrócić na drugą stronę, przy użyciu szerokiej
            łopatki.
    \end{enumerate}
    \newpage

    \subsection{Naleśniki}
    \bigskip
    \paragraph{Składniki:}
    \begin{wrapfigure}[1]{r}{0.4\textwidth}
        \includegraphics[width=0.4\textwidth]{nalesniki.jpg}
    \end{wrapfigure}
    \begin{enumerate}
        \item 1 szklanka mąki pszennej
        \item 2 jajka
        \item 1 szklanka mleka
        \item 3/4 szklanki wody (najlepiej gazowanej)
        \item szczypta soli
        \item 3 łyżki masła lub oleju roślinnego
    \end{enumerate}

    \paragraph{Przygotowanie}
    \begin{enumerate}
        \item Mąkę wsypać do miski, dodać jajka, mleko, wodę i sól. Zmiksować na
            gładkie ciasto.
        \item Dodać roztopione masło lub olej roślinny i razem zmiksować (lub
            wykorzystać tłuszcz do smarowania patelni przed smażeniem każdego
            naleśnika).
        \item Naleśniki smażyć na dobrze rozgrzanej patelni z cienkim dnem np.
            naleśnikowej. Przewrócić na drugą stronę gdy spód naleśnika będzie
            już ładnie zrumieniony i ścięty.
    \end{enumerate}
    \newpage

    % #######################################################################
    % #######################################################################
    % #######################################################################
    \section{Desery}
    \medskip
    \subsection{Ciasto czekoladowe}
    \bigskip

    \paragraph{Składniki:}
    \begin{wrapfigure}[3]{r}{0.4\textwidth}
        \includegraphics[width=0.4\textwidth]{ciasto_czekoladowe.jpg}
    \end{wrapfigure}
    \begin{enumerate}
        \item 80 g masła
        \item 100 g czekolady deserowej (lekko gorzkiej 50\%) lub po 50 g
            gorzkiej i mlecznej
        \item 1/2 szklanki (125 ml) mleka
        \item 2 jajka
        \item 150 g (3/4 szklanki) cukru
        \item 150 g (1 szklanka) mąki
        \item 1 łyżeczka proszku do pieczenia
    \end{enumerate}

    \paragraph{Przygotowanie}
    \begin{enumerate}
        \item Jajka ogrzać np. w misce z ciepłą wodą. Dno tortownicy o średnicy
            21 cm wyłożyć papierem do pieczenia, zapiąć obręcz. Piekarnik
            nagrzać do 175 stopni C (góra i dół).
        \item W rondelku umieścić pokrojone masło oraz połamaną na kosteczki
            czekoladę, podgrzewać na minimalnym ogniu ciągle mieszając aż
            składniki się roztopią i otrzymamy gładką masę czekoladową. Odstawić
            z ognia, dodać mleko i wymieszać na gładką masę.
        \item W misce ubić jajka z cukrem (ok. 4 minuty) na puszystą masę. Do
            drugiej miski przesiać mąkę, dodać proszek do pieczenia i dokładnie
            wymieszać. Dodać ubite jajka oraz masę czekoladową, zmiksować na
            minimalnych obrotach miksera lub wymieszać rózgą tylko do połączenia
            się składników w jednolite ciasto.
        \item Wyłożyć je do tortownicy, wstawić do piekarnika i piec przez ok.
            43 minuty do suchego patyczka (w większej niż 21 cm formie, ciasto
            będzie piekło się krócej, ok. 30 - 35 minut, sprawdzamy to
            patyczkiem wetkniętym w środek, czy jest jeszcze obklejony surowym
            ciastem).
    \end{enumerate}
    \newpage


\end{document}

